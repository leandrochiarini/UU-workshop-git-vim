\documentclass[reqno]{amsart}

\usepackage{amsmath,amsfonts,amscd,amssymb,amsthm,soul,color}
\usepackage[mathscr]{euscript}
\usepackage{ulem,cancel} 

\newtheorem{theorem}{Theorem}
\newtheorem{definition}[theorem]{Definition}
\newtheorem{conjecture}{Conjecture}
\newtheorem{corollary}[theorem]{Corollary}
\newtheorem{lemma}[theorem]{Lemma}
\newtheorem{proposition}[theorem]{Proposition}
\newtheorem{example}[theorem]{Example}
\newtheorem{remark}[theorem]{Remark}
\newtheorem{notation}[theorem]{Notation}
\newcommand{\mc}[1]{{\mathcal #1}}
\newcommand{\mf}[1]{{\mathfrak #1}}
\newcommand{\mb}[1]{{\mathbf #1}}
\newcommand{\bb}[1]{{\mathbb #1}}
\newcommand{\bs}[1]{{\boldsymbol #1}}
\newcommand{\ms}[1]{{\mathscr #1}}
\newcommand{\bbf}[1]{{\mathbf #1}}
\newcommand{\bl}[1]{{\color{blue} #1}}
\newcommand{\<}{\langle}
\renewcommand{\>}{\rangle}
\renewcommand{\>}{\rangle}

\title{Workshop for UU}
\author{}
\date{\today}


\begin{document}
\maketitle

\begin{theorem}[Fundamental theorem of Calculus]
  Let $f \in C^0([a,b])$, then consider the function
%
\begin{align*}
  \alpha(x):=
  \int_{a}^x f(y)  dy.
\end{align*}
% 
then $F^\prime(x)=f(x)$. 

\end{theorem}
\begin{proof}
\begin{align*}
  \alpha(x)
&:=
  \int_{a}^x f(y)  dy.
\end{align*}

\end{proof}


%%%%%%%%%%%%%%%%%%%%%%%%%%%%%%%%%%%%%%%%%%%%%%%%%%%%%%%%%%%%%%%%%%%
%Bibliography
%%%%%%%%%%%%%%%%%%%%%%%%%%%%%%%%%%%%%%%%%%%%%%%%%%%%%%%%%%%%%%%%%%%
%\bibliographystyle{abbrv}
%\bibliography{library}
\end{document}
